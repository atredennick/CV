
%______________________________________________________________________________________________________________________
% @brief    LaTeX2e Resume for Andrew T. Tredennick
\documentclass[margin,line]{resume}
\usepackage{url}
\usepackage{hyperref}
\pagestyle{plain}
\newcommand{\Cyclus}{\textsc{Cyclus }}
\usepackage[dvipsnames]{xcolor}
\usepackage{times}

\definecolor{mycol}{gray}{0.2}
\color{mycol}
%\usepackage{avant}
%\renewcommand{\familydefault}{\sfdefault}
%______________________________________________________________________________________________________________________
\begin{document}
\name{\color{MidnightBlue}{\textmd{\textsf{\LARGE{Andrew Tredennick}}}}}
\begin{resume}

    %__________________________________________________________________________________________________________________
    % Contact Information
    \section{\textmd{\textsf{\color{MidnightBlue}{Contact\\Information}}}}

    NSF Postdoctoral Fellow      							 \hfill Phone: (970) 443-1599            \vspace{0mm}\\\vspace{0mm}%
    Department of Wildland Resources and    		          \hfill E-mail: atredenn@gmail.com    \vspace{0mm}\\\vspace{0mm}%
    The Ecology Center, Utah State University      	          \hfill Web: \href{http://atredennick.github.io}{atredennick.github.io}  \vspace{0mm}\\\vspace{-4.5mm}%

    %__________________________________________________________________________________________________________________
    % Research Interests
    \section{\textmd{\textsf{\color{MidnightBlue}{Research\\Interests}}}}
		stability of populations, communities, and ecosystems; data--model assimilation; ecology of savannas and forests; coexistence of species and functional groups; ecological forecasting %
    %__________________________________________________________________________________________________________________
    % Academic Appointments
    %	 	\vspace{-2mm}
    %\section{\mysidestyle Academic\\Appointments}
    %           \vspace{-2mm}\\\vspace{-3mm}%
    %__________________________________________________________________________________________________________________
    % Education
    \section{\textmd{\textsf{\color{MidnightBlue}{Education}}}}

    \textbf{Colorado State University}, Fort Collins, CO \vspace{1mm}\\\vspace{1mm}%
    \textsl{Doctor of Philosophy} \textsc{Ecology}\hfill \textbf{2014}\vspace{-3mm}\\\vspace{-1mm}%
    
    \textbf{Texas Tech University}, Lubbock, TX \vspace{1mm}\\\vspace{1mm}%
    \textsl{Bachelor of Sciences} \textsc{Biology} \hfill \textbf{2006}\vspace{-3mm}\\\vspace{-1mm}%
    
     %__________________________________________________________________________________________________________________
    % Professional Experience
    \section{\textmd{\textsf{\color{MidnightBlue}{Professional\\Appointments}}}}
    \textbf{Utah State University, Dept. of Wildland Resources}, Logan, UT \hfill \textbf{2014 -- Pres.} \\
		\textsl{Postdoctoral Fellow} \vspace{.5mm}\\
    \textbf{Colorado State University, Natural Resource Ecology Lab}, Ft. Collins, CO \hfill \textbf{2011 -- 2014}\\
		\textsl{Graduate Research Fellow}\vspace{.5mm}\\ 
    \textbf{Colorado State University, Natural Resource Ecology Lab}, Ft. Collins, CO \hfill \textbf{2009 -- 2011}\\
		\textsl{Graduate Research Assistant}\vspace{.5mm}\\ 
     \textbf{US Forest Service Rocky Mountain Research Station}, Ft. Collins, CO \hfill \textbf{2009}\\
		\textsl{Research Assistant}\vspace{.5mm}\\ 		
     \textbf{Colorado State University}, Ft. Collins, CO \hfill \textbf{2008}\\
		\textsl{Graduate Teaching Assistant}\\ 

    %__________________________________________________________________________________________________________________
    % Honors and Awards
     \section{\textmd{\textsf{\color{MidnightBlue}{Fellowships \\and Awards}}}}
        		NEON and Powell Center Travel Award \hfill \textbf{2015}\vspace{.5mm}\\%
                NSF Postdoctoral Research Fellowship in Biology and Mathematics    \hfill \textbf{2014}\vspace{.5mm}\\%
                1\textsuperscript{st} Place Oral Presentation, Front Range Student Ecology Symposium  \hfill \textbf{2013}\vspace{.5mm}\\%
                NSF FORECAST Research Coordination Network Travel Award \hfill \textbf{2012}\vspace{.5mm}\\%
                Sustainability Leadership Fellow, School of Global Environmental Sustainability, CSU \hfill \textbf{2012}\vspace{.5mm}\\%
		NASA Earth and Space Science Graduate Fellowship  \hfill \textbf{2011}\vspace{.5mm}\\%
		James E. Ellis Memorial Scholarship, Natural Resource Ecology Lab  \hfill \textbf{2010}\vspace{.5mm}\\%
		NSF Graduate K-12 Fellowship, Natural Resource Ecology Lab                        \hfill\textbf{2009}%
   
   %__________________________________________________________________________________________________________________
    % Refereed Journal Publications
      \section{\textmd{\textsf{\color{MidnightBlue}{Publications}}}}
	
	  \textbf{Tredennick, A.T.}, M.B. Hooten,  and P.B. Adler. (In press). Do we need demographic data to forecast plant population dynamics? \emph{Methods in Ecology and Evolution}.
	
	  \textbf{Tredennick, A.T.}, M.B. Hooten, C.L. Aldridge, C. Homer, A.R. Kleinhesselink, and P.B. Adler. (In press). Forecasting climate change impacts on plant populations over large spatial extents. \emph{Ecosphere}.
	
	\textbf{Tredennick, A.T.}, P.B. Adler, J.B. Grace, W.S. Harpole, E.T. Borer, E.W. Seabloom, and 36 co-authors. (2016). Comment on ``Worldwide evidence of a unimodal relationship between productivity and plant species richness". \textsl{Science} 35(6272):457a-457c. \vspace{-6mm}\\%
	
	\textbf{Tredennick, A.T.}, M. Karemb\'{e}, F. Demb\'{e}l\'{e}, J. Dohn, and N.P Hanan. No effects of fire, large herbivores, and their interaction on regrowth of harvested trees in two West African savannas. \textsl{African Journal of Ecology} 53(4):487-495. \vspace{-6mm}\\%
	
	Hanan, N. P., \textbf{A.T. Tredennick}, L. Prihodko, G. Bucini, and J. Dohn. (2015). Analysis of stable states in global savannas -- a response to Staver and Hansen. 	\textsl{Global Ecology and Biogeography} 24(8):988-989. \vspace{-6mm}\\%
   
     	\textbf{Tredennick, A.T.} and N.P. Hanan. (2015). Effects of tree harvest on the stable-state dynamics of savanna and forest. \textsl{The American Naturalist} 5(185):E153-E165. \vspace{-6mm} \\%

	\newpage{}
	
	Hanan, N. P., \textbf{A.T. Tredennick}, L. Prihodko, G. Bucini, and J. Dohn. (2014). Analysis of stable states in global savannas: Is the CART pulling the horse? 	\textsl{Global Ecology and Biogeography} 23(3):259-263. \vspace{-6mm}\\%
	
	\textbf{Tredennick, A. T.}, L.P. Bentley, and N.P. Hanan. (2013). Allometric convergence in savanna trees and implications for the use of plant scaling models in 	variable ecosystems. \textsl{PLoS One} 8(3):e58241. \vspace{-6mm}\\%

   Rice, J., \textbf{A.T. Tredennick}, and L. Joyce. (2011). The climate of the Shoshone National Forest: A synthesis of past changes, future projections, and 			ecosystem implications. USDA National Forest Service General Technical Report 264. \vspace{-6mm}\\%
	
	Sutton, A.E., J. Dohn, K. Loyd, \textbf{A.T. Tredennick}, G. Bucini, A. Sol�rzano, L. Prihodko, and N.P. Hanan. (2010). Letter: Does warming increase the risk of civil war 	in Africa? \textsl{Proceedings of the National Academy of Sciences} 107(25):E102 \\ 
		
	\section{\textmd{\textsf{\color{MidnightBlue}{Publications \\ \footnotesize{in review}}}}}
	%\section{\mysidestyle Publications\\ \textsl{\footnotesize in review}}
     \textbf{Tredennick, A.T.}, C. de Mazancourt, M. Loreau, and P.B. Adler. Environmental responses, not species interactions, determine species synchrony in natural plant communities. In review at \emph{Ecology}.
	
	Kulmatiski, A., P.B. Adler, J.M. Stark, and \textbf{A.T. Tredennick}. Water and nitrogen uptake are better associated with resource availability than root biomass. In review at \emph{Nature Ecology and Evolution}.
	
	\section{\textmd{\textsf{\color{MidnightBlue}{Manuscripts \\ \footnotesize{in preparation} \\ \footnotesize{(available upon request)}}}}}
	 %\section{\mysidestyle Manuscripts\\ \textsl{\footnotesize in preparation}\\ \textsl{\footnotesize (available upon request)}}  
    Wilcox*, K.R., \textbf{A.T. Tredennick}*, S.E. Koerner, E. Grman, L.M. Hallett, M.L. Avolio, K.J. La Pierre, G.R. Houseman, F. Isbell, D.S. Johnson, and S. Baer. Asynchrony of ecosystem functioning across space increases ecosystem stability through time. In preparation for November submission to \emph{Ecology Letters}. \\%
    \hspace{2em} \textsf{\footnotesize{*Shared first authorship by Wilcox and Tredennick}} \vspace{-6mm}\\%
    
    \textbf{Tredennick, A.T.}, P.B. Adler, and F.R. Adler. Revisiting how species coexist reveals a negative diversity-stability relationship. In preparation for December submission to \emph{Ecology Letters}.
    
	
    %__________________________________________________________________________________________________________________
    % Funding
    \section{\textmd{\textsf{\color{MidnightBlue}{Competitive\\Funding \\ \footnotesize{External: \$297,000} \\ \footnotesize{Internal: \$17,500}}}}}
    %\section{\mysidestyle Competitive\\Funding\\\textsl{\footnotesize External: \$297,000}\\\textsl{\footnotesize Internal: \$17,500}}
    ``Diversity-Stability Relationships and Coexistence: New Theory and Empirical Tests," NSF Postdoctoral Research Fellowship in Biology and Mathematics, \$207,000 (2014-2017)\vspace{2mm}\\%
    ``Effective Science Communication and Public Relations at NREL through EcoPress," Natural Resource Ecology Lab, \$6,000 (Co-I) (2013) \vspace{2mm}\\%
    ``NESSF: Fuelwood, Savannas, and Climate Change: Integrating Modeling, Field Experimentation, and Optical and Radar Remote Sensing," NASA, \$90,000 (2011-2014) \vspace{2mm}\\%
    ``Expanding Ecology to Meet Society: Traditional Experiments Coupled with Anthropological Methods in a Savanna Socio-Ecological System,'' Natural Resource Ecology Laboratory James E. Ellis Memorial Scholarship, \$1,500 (2010)\vspace{2mm}\\%
    ``Building a WCNR `Partnership for International Research and Education' in African Savannas: Undergraduate and Graduate Field-Based Education in Mali, West Africa," Warner College of Natural Resources, \$10,000 (PI--Hanan; collaborative proposal of the Hanan lab group) (2010)%
    
     %__________________________________________________________________________________________________________________
    % Data products
    %\section{\textmd{\textsf{\color{MidnightBlue}{Data\\Products}}}}
   % \section{\mysidestyle Data\\Products}
    %Tredennick AT, Bentley LP, Hanan NP (2013) Data from: Allometric convergence in savanna trees and implications for plant scaling models in variable ecosystems. Dryad Digital Repository. \\ http://dx.doi.org/10.5061/dryad.4s1d2 \vspace{-6mm}\\%
    
    %Tredennick AT, Hanan NP, Martinez K, Keita, L (2014) Data from: Effects of tree harvest on the stable-state dynamics of savanna and forest. Dryad Digital Repository. \\ http://dx.doi.org/10.5061/dryad.vg121
    
    %__________________________________________________________________________________________________________________
    % Teaching Experience
     \section{\textmd{\textsf{\color{MidnightBlue}{Teaching\\Experience}}}}
   % \section{\mysidestyle Teaching\\Experience}
    \textbf{Ecological Society of America Annual Meetings Workshop} \hfill \textbf{2013-2016}\\
    		\textsl{Data Visualization in R}\\
		Co-organizer and co-instructor (with Naupaka Zimmermann)\\
		Materials: http://github.com/atredennickesa\_data\_viz\_2016
		

    
    \textbf{Colorado State University} \hfill \textbf{2013}\\
               \textsl{Plant Ecology (undergraduate)}\\
               Guest Lecture on Tree-Grass Coexistence
               
    \textbf{Colorado State University} \hfill \textbf{2012}\\ 
               \textsl{NREL Skills for Undergraduate Participation in Ecological Research}\\
               Data Analysis/Visualization Workshop Leader
               		      \newpage{}
    \textbf{Colorado State University} \hfill \textbf{2011}\\ 
               \textsl{RS 351, Ecosystem Processes in a Changing World (undergraduate)}\\
               Co-Instructor 

    \textbf{Colorado State University} \hfill \textbf{2009, 2010, 2012}\\ 
               \textsl{RS 351, Wildland Ecosystems (undergraduate)}\\
               Guest Lecture on Ecosystem Modeling
               
    \textbf{Irish Elementary School} Ft. Collins, CO \hfill \textbf{2010-2011}\\ 
   		\textsl{5th and 4th Grade Science and Advanced Science Program}\\
		NSF GK-12 Fellow
		
     \textbf{Colorado State University} \hfill \textbf{2008}\\ 
               \textsl{RS 351, Wildland Ecosystems (undergraduate)}\\
               Graduate Teaching Assistant 
%__________________________________________________________________________________________________________________
    % Service
     \section{\textmd{\textsf{\color{MidnightBlue}{Professional\\Service}}}}
    %\section{\mysidestyle Professional\\Service}
    \textbf{Reviewer}\\
    	\textsl{Proceedings of the National Academy of Sciences};
	\textsl{Ecology Letters};
  \textsl{Ecology};
	\textsl{Ecological Applications};
	\textsl{Journal of Ecology};
	\textsl{Oecologia};
  \textsl{Elementa};
  \textsl{PLoS One};
	\textsl{Forest Ecology and Management};
	\textsl{Agriculture, Ecosystems, and Environment};
	\textsl{Environmental Management};
	 \textsl{Koedoe: African Protected Area Conservation and Science};
	National Research Foundation (South Africa);
	National Science Foundation (\emph{ad hoc} reviewer)
	

 \textbf{Professional Society Membership}\\
    	Ecological Society of America\\
	American Society of Naturalists

%\section{\mysidestyle Professional\\Service\\ \textsl{\footnotesize continued}}

   \textbf{University Service and Public Outreach}\\
   	Director of Social Media, \textsl{NREL EcoPress}: http://nrelscience.org (2012-2013)\\
	Student Review Committee, Ecology Faculty Search, CSU (2011)\\
	Organizing Committee, Global Savanna Workshop, CSU (2009)\\
	Advertising and Outreach Committee, Front Range Student Ecology Symposium (2010)\\
	Graduate Student Representative, Natural Resource Ecology Lab (2010-2011)
  
%__________________________________________________________________________________________________________________
    % Presentations
     \section{\textmd{\textsf{\color{MidnightBlue}{Presentations}}}}
   % \section{\mysidestyle Presentations}
        Tredennick, A.T. C. de Mazancourt, M. Loreau, and P.B. Adler. (2016) ``Disentangling the drivers of species synchrony in natural plant communities: Environmental forcing, demographic stochasticity, and interspecific interactions''. 2016 Annual Meetings of the Ecological Society of America.\vspace{-6mm}\\%
        
    Tredennick, A.T. and P.B. Adler. (2015) ``Do we need detailed demographic data to forecast population responses to climate change? 2015 Annual Meetings of the Ecological Society of America.\vspace{-6mm}\\%
    
    Tredennick, A.T., Hanan, N.P., Bucini, G., and Parton, W. (2014) ``Africa's Fuelwood Footprint and the Biome-Level Impacts of Tree Harvest,'' Station d'Ecologie Exp\'{e}rimentale du CNRS, Moulis, France. \vspace{-6mm}\\% 
    
    Tredennick, A.T., Adler, P.B., Aldridge, C.L., Homer C.G., Iles, D.T., Kleinhesselink, A., LaMalfa, E., and Mann, R. (2014) ``Forecasting climate change impacts on plant population dynamics at large spatial extents: a test case with sagebrush (\emph{Artemisia}) species.'' 2014 Annual Meetings of the Ecological Society of America.\vspace{-6mm}\\%
    
    Tredennick, A.T., Adler, P.B., Aldridge, C.L., Homer C.G., Iles, D.T., Kleinhesselink, A., LaMalfa, E., and Mann, R. (2013) ``Pixel-based modeling of population dynamics at large spatial extents,'' Climate Change in Sagebrush Steppe Working Group, Park City, UT. \vspace{-6mm}\\%
    

    Tredennick, A.T., Hanan, N.P., Bucini, G., and Parton, W. (2013) ``Sustainability and Biome-Level Impacts of Fuelwood Harvesting in sub-Saharan Africa,'' invited seminar, Geospatial Science Center of Excellence, South Dakota State University. \vspace{-6mm}\\%
    
    Tredennick, A.T., and Hanan N.P. (2013) ``Tree harvest, fire, and drought can drive state transitions in savanna,'' 2013 Annual Meetings of the Ecological Society of America. \vspace{-6mm}\\%
            \newpage{}    
    
    Tredennick, A.T., and Hanan N.P. (2013) ``The Theoretical and Integrative Effects of Tree Harvest and Fire on Grassland-Savanna-Forest Transitions,'' Front Range Student Ecology Symposium, Colorado State University. \textsl{Awarded 1\textsuperscript{st} Place Oral Presentation}. \vspace{-6mm}\\%
    
    Tredennick, A.T., Hanan, N.P., Bucini, G., and Parton, W. (2012) ``Using Diverse Multi-Scale Data to Assess Patterns and Sustainability of Fuelwood Harvest in Sub-Saharan Africa,'' poster presentation, NSF FORECAST RCN Meeting, Woods Hole, MA. \vspace{-6mm}\\%
    
    Tredennick, A.T., Bentley, L.P., and Hanan, N.P. (2012) ``Whole-tree and branch-level scaling in savannas: testing Metabolic Scaling Theory in a non-ideal system,'' Ecological Society of America Annual Meeting, Portland, OR. \vspace{-6mm}\\%
    
    Tredennick, A.T., Hanan, N.P., Bucini, G., and Parton, W. (2012) ``Patterns and sustainability of fuelwood supply and demand in Sub-Saharan Africa,'' 10th Annual Savanna Science Network Meeting, Kruger National Park, South Africa.\vspace{-6mm}\\%
    
    Tredennick, A.T., Hanan, N.P., and Bentley, L.P. (2012) ``Scaling the savannas: Does Metabolic Scaling Theory apply in savannas?,'' Poster Presentation, 10th Annual Savanna Science Network Meeting, Kruger National Park, South Africa. \vspace{-6mm}\\%

    Tredennick, A.T., Hanan, N.P., Bucini, G., Parton, W., and Keogh, C. (2011) ``Spatially Quantifying Fuelwood Demand and Production in Sub-Saharan Africa,'' NREL Spring Seminar Series: New Voices in Ecology, Colorado State University.\vspace{-6mm}\\%
    
    Tredennick, A.T. and Hanan N.P. (2011) ``Allometric Scaling in Savannas: Do the Non-conformists Conform to Ecological Theory?,'' Front Range Student Ecology Symposium, Colorado State University. \textsl{Awarded 2\textsuperscript{nd} Place Oral Presentation}.\vspace{-6mm}\\%
  
  Tredennick, A.T. and Coughenour, M.B. (2009) ``Economic Incentives for Conservation in Meru, Kenya,'' Poster Presentation, Front Range Student Ecology Symposium, Colorado State University.

%______________________________________________________________________________________________________________________
\end{resume}
\end{document}


%______________________________________________________________________________________________________________________
% EOF

